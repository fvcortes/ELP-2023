% 2022 University of Pennsylvania PhD Dissertation LaTeX Template v.20221216
% This template aims to fulfill University formatting requirements when one follows the instructions, resolves errors and warnings, and does not edit the style file or font size. Consult https://provost.upenn.edu/formatting-faqs for the most up-to-date requirements and policies.
%-------------------------------------------
% We welcome your feedback on the template https://upenn.libwizard.com/f/dissertationlatextemplatefeedback
%-------------------------------------------
% The template requires TeX Live 2022 and is optimized for pdfLaTeX and the LaTeX editor Overleaf. Other LaTeX editing programs may require you to compile the document more than once. It is adapted from the Penn Biostat LaTeX Template modified from Dissertation Template for Wharton PhD Candidates in LaTeX. Portions of the text are reprinted or adapted with permission from Ratcliffe SJ. (2017) Penn Biostat LaTeX Template: PhD dissertation. https://dbe.med.upenn.edu/biostat-research/Dissertation_template

%%%%%%%%%%%%%%%%%%%%%%%%%%%%%%%%%%%%%%%%%%%%%%%%
%%%%%%%%%%%%%%%%%%%%%%%%%%%%%%%%%%%%%%%%%%%%%%%%
\documentclass[11pt]{report}
\usepackage{upennstyle} % You may add additional packages below this line. Overleaf will warn you if your packages conflict with the template packages. See Chapter 1 for more information.

%-------------------------------------------
% Citation and bibliography commands. Edit as needed.
\usepackage[nonamebreak,round]{natbib}
\bibliographystyle{plainnat}
%-------------------------------------------

%%%%%%%%%%%%%%%%%%%%%%%%%%%%%%%%%%%%%%%%%%%%%%%%
%%%%%%%%%%%%%%%%%%%%%%%%%%%%%%%%%%%%%%%%%%%%%%%%
% PLEASE FOLLOW THE INSTRUCTIONS IN THE CODE COMMENTS
% To uncomment a line of code, remove the % at the beginning of the line
% To comment out a line of code, add a % at the beginning of the line
%%%%%%%%%%%%%%%%%%%%%%%%%%%%%%%%%%%%%%%%%%%%%%%%
%%%%%%%%%%%%%%%%%%%%%%%%%%%%%%%%%%%%%%%%%%%%%%%%

%%%%% PRELIMINARY PAGES %%%%%
%-------------------------------------------
%%%% TITLE PAGE
% Edit the text between curly braces {} relevant to your dissertation.
\title{REDUCING IMPACTS OF CLIMATE CHANGE}
\author{Júlia Magalhães de Sant'Anna}
%\specialization{Field of Specialization} % If you are in Romance Languages or Managerial Science and Applied Economics (Wharton), uncomment this line and edit the text between the {}. If you do, you will also need to edit lines 71 and 72.
\gradgroup{English Learning Program} % See https://provost.upenn.edu/phd-graduate-groups for names
\date{2023} % Enter current four-digit year
% \supervisor{Supervisor's Typed Name}
% \supervisortitle{Full Faculty Title and Affiliation}
%\cosupervisor{Co-supervisor's Typed Name} % If applicable, uncomment this line and edit the text between the {}. If you do, you will also need to edit line 39. 
%\cosupervisortitle{Full Faculty Title and Affiliation} % If applicable, uncomment this line and edit the text between the {}. If you do, you will also need to edit line 38.
% \gradchair{Graduate Group Chairperson's Typed Name, Full Faculty Title}
% \committee{Committee Member's Typed Name, Full Faculty Title and Affiliation}
% \committee{Committee Member's Typed Name, Full Faculty Title and Affiliation}
% \committee{Committee Member's Typed Name, Full Faculty Title and Affiliation} % If applicable, comment out or duplicate this line to remove or add a committee member
%-------------------------------------------

%%%% OPTIONAL COPYRIGHT NOTICE
% \authorlegal{Author’s Full Legal Name} % To opt out of copyrighting your dissertation, uncomment this line. If you do, you will also need to edit line 76.
%-------------------------------------------
%\cclicense{Name of Selected Creative Commons License} % If applicable, uncomment this line and edit the text between the {}. If you do, you will also need to edit lines 51, 76, and 78.
%-------------------------------------------
%\cclicenseurl{URL of selected Creative Commons License} % If applicable, uncomment this line and edit the text between the {}. If you do, you will also need to edit lines 49, 76, and 78.
%-------------------------------------------

%%%% OPTIONAL DEDICATION
% \dedication{Write your dedication text in italics here.} % If not applicable, comment out this line to hide the optional dedication page. If you do, you will also need to edit line 82.
%-------------------------------------------

%%%% OPTIONAL ACKNOWLEDGEMENT
% \acknowledgement{Write your acknowledgement text here.} % If not applicable, comment out this line to hide the optional acknowledgement page. If you do, you will also need to edit line 86.
%-------------------------------------------

%%%% ABSTRACT
% \abstract{Write your abstract text here.} % Write your abstract between the {}
%-------------------------------------------

%%%% OPTIONAL PREFACE
%\preface{Write your preface text here.} % If applicable, uncomment this line and write your preface between the {}. If you do, you will also need to edit line 101.
%-------------------------------------------

\begin{document}
\maketitle % If you are in Romance Languages or Managerial Science and Applied Economics (Wharton), comment out this line. If you do, you will also need to edit lines 33 and 72.
%\makespecializationtitle % If you are in Romance Languages or Managerial Science and Applied Economics (Wharton), uncomment this line. If you do, you will also need to edit lines 33 and 71.
\setcounter{page}{2}

%%%% OPTIONAL COPYRIGHT NOTICE
% \makecopyright % If not applicable, comment out this line to hide the optional traditional copyright notice page. If you do, you will also need to edit line 47.
%-------------------------------------------
%\makecreativecommons % If applicable, uncomment this line to insert the optional Creative Commons License copyright notice page. If you do, you will also need to edit lines 49, 51, and 76.
%-------------------------------------------

%%%% OPTIONAL DEDICATION PAGE
% \makededication % If not applicable, comment out this line to hide the optional dedication page. If you do, you will also need to edit line 55.
%-------------------------------------------

%%%% OPTIONAL ACKNOWLEDGEMENT PAGE
% \makeacknowledgement % If not applicable, comment out this line to hide the optional acknowledgment page. If you do, you will also need to edit line 59.
%-------------------------------------------

% \makeabstract
% \tableofcontents

%%%% OPTIONAL LIST OF TABLES
% \clearpage \phantomsection \addcontentsline{toc}{chapter}{LIST OF TABLES} \begin{singlespacing} \listoftables \end{singlespacing}% If not applicable, comment out this line to hide the optional List of Tables
%-------------------------------------------

%%%% OPTIONAL LIST OF ILLUSTRATIONS
% \clearpage \phantomsection \addcontentsline{toc}{chapter}{LIST OF ILLUSTRATIONS} \begin{singlespacing} \listoffigures \end{singlespacing}% If not applicable, comment out this line to hide the optional List of Illustrations
%-------------------------------------------

%%%% OPTIONAL PREFACE
%\makepreface % If applicable, uncomment this line to insert the optional preface. If you do, you will also need to edit line 67.
%-------------------------------------------

%%%%% MAIN TEXT %%%%%
\begin{mainf} % The main body of your dissertation starts below this line
\chapter{Unit 5}
\section*{Assessment 2}
    Climate change is everywhere \cite{behrens_lemuridae_2016}.
    This is a statement, \cite{lake_white_1979}.
    
   
    Scientists have the role response in leading a movement in the development of "climate technologies". To make it possible, scientific community needs to share their local information about climate change and their multidisciplinar points of view. To promote this connections, Universities and schools should promote spaces to conferences with the enviromental theme. In order to make this movement popular and widly discussed.

    The real solutions to the climate change must come from differents scientific areas working together to achive this goal.  The roots of the solutions that we are looking for may consist in a scientific group awareness about the planet's health.
     
    Academic projects like the ELP english program for STEM is an example of a program which provides a connection with different scientific areas and guides through to the awareness about the climate changes.

    The climate technologies can be partitionated in two general concepts. Hard climate technologies and Soft climate technologies. Hard climate technologies are related to technologies such as wind power, sun power and water power development, for example. Soft climate technologies are described as the economy of energy, training and correct usage of energy and resources in the society.

    With all these in mind, future might hold scientific community generation that imports about science and it's usage focus on such important topics as the ocean pollution, new renewable energy and efficiency to produce food and suplies to the earth population.
    brillant new discoveries that maybe change it's path to a health and ballanced technological world.

    



    [This technologies are meant to promote the discussion and development of the need to change the way people think
    These technologies are mean to increase the discussion focus about hard and soft technologies,
    Climate technologies are described as (description and examples)
    The scientific comunity must exchenge knowlegde, share informations and brainstorm about the climate changes topics.
    The solution for climate change won't come from a single source, it is a joint effort that must come from all areas.
    Group awareness might have a great impact in reducing climate change.]



bibliography in course 

https://unfccc.int/ttclear/misc_/StaticFiles/gnwoerk_static/NAD_EBG/54b3b39e25b84f96aeada52180215ade/b8ce50e79b574690886602169f4f479b.pdf

https://www.oecd.org/science/the-contribution-of-science-and-innovation-to-addressing-climate-change.htm#:~:text=Science%20underpins%20the%20global%20consensus,in%20phasing%20out%20polluting%20ones.


\end{mainf}
%-------------------------------------------

%%%%% OPTIONAL APPENDICES %%%%%
%-------------------------------------------
% \begin{append}
% \chapter{\MakeUppercase{Title of Appendix A}}
% The content of Appendix A begins here. Use the \verb|\chapter| command to insert additional appendices.
% \section{Section Name of Appendix A}
% Use the \verb|\section| command to create sections.
% \chapter{\MakeUppercase{Title of Appendix B}}
% The content of Appendix B begins here.
% \end{append}
%-------------------------------------------

%%%%% BIBLIOGRAPHY %%%%%
%-------------------------------------------
\begin{bibliof}
%\nocite{*} % If applicable, uncomment this line to display all entries in the .bib file
\bibliography{bibliography}
\end{bibliof}
%-------------------------------------------
\end{document}
